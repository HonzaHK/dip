%==============================================================================
% tento soubor pouzijte jako zaklad
% this file should be used as a base for the thesis
% Autoři / Authors: 2008 Michal Bidlo, 2016 Jaroslav Dytrych
% Kontakt pro dotazy a připomínky: dytrych@fit.vutbr.cz
% Contact for questions and comments: dytrych@fit.vutbr.cz
%==============================================================================
% kodovani: UTF-8 (zmena prikazem iconv, recode nebo cstocs)
% encoding: UTF-8 (you can change it by command iconv, recode or cstocs)
%------------------------------------------------------------------------------
% zpracování / processing: make, make pdf, make clean
%==============================================================================
% Soubory, které je nutné upravit: / Files which have to be edited:
%   projekt-20-literatura-bibliography.bib - literatura / bibliography
%   projekt-01-kapitoly-chapters.tex - obsah práce / the thesis content
%   projekt-30-prilohy-appendices.tex - přílohy / appendices
%==============================================================================
\documentclass[]{fitthesis} % bez zadání - pro začátek práce, aby nebyl problém s překladem
%\documentclass[english]{fitthesis} % without assignment - for the work start to avoid compilation problem
%\documentclass[zadani]{fitthesis} % odevzdani do wisu - odkazy jsou barevné
%\documentclass[english,zadani]{fitthesis} % for submission to the IS FIT - links are color
%\documentclass[zadani,print]{fitthesis} % pro tisk - odkazy jsou černé
%\documentclass[zadani,cprint]{fitthesis} % pro barevný tisk - odkazy jsou černé, znak VUT barevný
%\documentclass[english,zadani,print]{fitthesis} % for the color print - links are black
%\documentclass[english,zadani,cprint]{fitthesis} % for the print - links are black, logo is color
% * Je-li práce psaná v anglickém jazyce, je zapotřebí u třídy použít 
%   parametr english následovně:
%   If thesis is written in english, it is necessary to use 
%   parameter english as follows:
%      \documentclass[english]{fitthesis}
% * Je-li práce psaná ve slovenském jazyce, je zapotřebí u třídy použít 
%   parametr slovak následovně:
%   If the work is written in the Slovak language, it is necessary 
%   to use parameter slovak as follows:
%      \documentclass[slovak]{fitthesis}
% * Je-li práce psaná v anglickém jazyce se slovenským abstraktem apod., 
%   je zapotřebí u třídy použít parametry english a enslovak následovně:
%   If the work is written in English with the Slovak abstract, etc., 
%   it is necessary to use parameters english and enslovak as follows:
%      \documentclass[english,enslovak]{fitthesis}

% Základní balíčky jsou dole v souboru šablony fitthesis.cls
% Basic packages are at the bottom of template file fitthesis.cls
% zde můžeme vložit vlastní balíčky / you can place own packages here

% Kompilace po částech (rychlejší, ale v náhledu nemusí být vše aktuální)
% Compilation piecewise (faster, but not all parts in preview will be up-to-date)
% \usepackage{subfiles}

% Nastavení cesty k obrázkům
% Setting of a path to the pictures
%\graphicspath{{obrazky-figures/}{./obrazky-figures/}}
%\graphicspath{{obrazky-figures/}{../obrazky-figures/}}

%---rm---------------
\renewcommand{\rmdefault}{lmr}%zavede Latin Modern Roman jako rm / set Latin Modern Roman as rm
%---sf---------------
\renewcommand{\sfdefault}{qhv}%zavede TeX Gyre Heros jako sf
%---tt------------
\renewcommand{\ttdefault}{lmtt}% zavede Latin Modern tt jako tt

% vypne funkci šablony, která automaticky nahrazuje uvozovky,
% aby nebyly prováděny nevhodné náhrady v popisech API apod.
% disables function of the template which replaces quotation marks
% to avoid unnecessary replacements in the API descriptions etc.
\csdoublequotesoff

% =======================================================================
% balíček "hyperref" vytváří klikací odkazy v pdf, pokud tedy použijeme pdflatex
% problém je, že balíček hyperref musí být uveden jako poslední, takže nemůže
% být v šabloně
% "hyperref" package create clickable links in pdf if you are using pdflatex.
% Problem is that this package have to be introduced as the last one so it 
% can not be placed in the template file.
\ifWis
\ifx\pdfoutput\undefined % nejedeme pod pdflatexem / we are not using pdflatex
\else
  \usepackage{color}
  \usepackage[unicode,colorlinks,hyperindex,plainpages=false,pdftex]{hyperref}
  \definecolor{links}{rgb}{0.4,0.5,0}
  \definecolor{anchors}{rgb}{1,0,0}
  \def\AnchorColor{anchors}
  \def\LinkColor{links}
  \def\pdfBorderAttrs{/Border [0 0 0] }  % bez okrajů kolem odkazů / without margins around links
  \pdfcompresslevel=9
\fi
\else % pro tisk budou odkazy, na které se dá klikat, černé / for the print clickable links will be black
\ifx\pdfoutput\undefined % nejedeme pod pdflatexem / we are not using pdflatex
\else
  \usepackage{color}
  \usepackage[unicode,colorlinks,hyperindex,plainpages=false,pdftex,urlcolor=black,linkcolor=black,citecolor=black]{hyperref}
  \definecolor{links}{rgb}{0,0,0}
  \definecolor{anchors}{rgb}{0,0,0}
  \def\AnchorColor{anchors}
  \def\LinkColor{links}
  \def\pdfBorderAttrs{/Border [0 0 0] } % bez okrajů kolem odkazů / without margins around links
  \pdfcompresslevel=9
\fi
\fi
% Řešení problému, kdy klikací odkazy na obrázky vedou za obrázek
% This solves the problems with links which leads after the picture
\usepackage[all]{hypcap}

% Informace o práci/projektu / Information about the thesis
%---------------------------------------------------------------------------
\projectinfo{
  %Prace / Thesis
  project={SP},            %typ práce BP/SP/DP/DR  / thesis type (SP = term project)
  year={2018},             % rok odevzdání / year of submission
  date=\today,             % datum odevzdání / submission date
  %Nazev prace / thesis title
  title.cs={Nástroj pro podporu provádění analýzy finančního zdraví firmy},  % název práce v češtině či slovenštině (dle zadání) / thesis title in czech language (according to assignment)
  title.en={Tool for Execution of a Firm's Financial Healthiness Analysis}, % název práce v angličtině / thesis title in english
  %title.length={14.5cm}, % nastavení délky bloku s titulkem pro úpravu zalomení řádku (lze definovat zde nebo níže) / setting the length of a block with a thesis title for adjusting a line break (can be defined here or below)
  %Autor / Author
  author.name={Jan},   % jméno autora / author name
  author.surname={Kubiš},   % příjmení autora / author surname 
  author.title.p={Bc.}, % titul před jménem (nepovinné) / title before the name (optional)
  %author.title.a={Ph.D.}, % titul za jménem (nepovinné) / title after the name (optional)
  %Ustav / Department
  department={UIFS}, % doplňte příslušnou zkratku dle ústavu na zadání: UPSY/UIFS/UITS/UPGM / fill in appropriate abbreviation of the department according to assignment: UPSY/UIFS/UITS/UPGM
  % Školitel / supervisor
  supervisor.name={Šárka},   % jméno školitele / supervisor name 
  supervisor.surname={Květoňová},   % příjmení školitele / supervisor surname
  supervisor.title.p={Ing.},   %titul před jménem (nepovinné) / title before the name (optional)
  supervisor.title.a={Ph.D.},    %titul za jménem (nepovinné) / title after the name (optional)
  % Klíčová slova / keywords
  keywords.cs={finanční analýza, hodnocení finančního zdraví}, % klíčová slova v českém či slovenském jazyce / keywords in czech or slovak language
  keywords.en={financial analysis, financial healthiness rating}, % klíčová slova v anglickém jazyce / keywords in english
  % Abstrakt / Abstract
  abstract.cs={Tématem této práce je finanční analýza podniku a popis programového řešení pro její provádění. V první části je popsána finanční analýza podniku obecně -- je zde shrnuta motivace, množina vstupních dat, výčet používaných metod a ukazatelů, jejich význam a interpretace. Druhá část práce obsahuje popis návrhu, implementace a fungování nástroje, který je schopen provádět vybrané výpočty z finanční analýzy}, % abstrakt v českém či slovenském jazyce / abstract in czech or slovak language
  abstract.en={Topic of this ? is a company financial analalysis and description of software for its purpose. In the first part, there is description of company financial analysis in general -- motivation, input data set, summary of common methods and indicators, their meaning and iterpretation. The second part contains description of design, implementation and functionality of a tool, which is able to perform chosen ? from financial analysis.}, % abstrakt v anglickém jazyce / abstract in english
  % Prohlášení (u anglicky psané práce anglicky, u slovensky psané práce slovensky) / Declaration (for thesis in english should be in english)
  declaration={Prohlašuji, že jsem tuto práci vypracoval samostatně pod vedením Ing. Šárky \\Květoňové, Ph.D.. Uvedl jsem všechny literární prameny a publikace, ze kterých jsem čerpal.},
  %declaration={Hereby I declare that this bachelor's thesis was prepared as an original author’s work under the supervision of Mr. X
% The supplementary information was provided by Mr. Y
% All the relevant information sources, which were used during preparation of this thesis, are properly cited and included in the list of references.},
  % Poděkování (nepovinné, nejlépe v jazyce práce) / Acknowledgement (optional, ideally in the language of the thesis)
  acknowledgment={Děkuji Ing. Šárce Květoňové, Ph.D. za vedení mé diplomové práce.},
  %acknowledgment={Here it is possible to express thanks to the supervisor and to the people which provided professional help
%(external submitter, consultant, etc.).},
  % Rozšířený abstrakt (cca 3 normostrany) - lze definovat zde nebo níže / Extended abstract (approximately 3 standard pages) - can be defined here or below
  %extendedabstract={Do tohoto odstavce bude zapsán rozšířený výtah (abstrakt) práce v českém (slovenském) jazyce.},
  %faculty={FIT}, % FIT/FEKT/FSI/FA/FCH/FP/FAST/FAVU/USI/DEF
  faculty.cs={Fakulta informačních technologií}, % Fakulta v češtině - pro využití této položky výše zvolte fakultu DEF / Faculty in Czech - for use of this entry select DEF above
  faculty.en={Faculty of Information Technology}, % Fakulta v angličtině - pro využití této položky výše zvolte fakultu DEF / Faculty in English - for use of this entry select DEF above
  department.cs={Ústav matematiky}, % Ústav v češtině - pro využití této položky výše zvolte ústav DEF nebo jej zakomentujte / Department in Czech - for use of this entry select DEF above or comment it out
  department.en={Institute of Mathematics} % Ústav v angličtině - pro využití této položky výše zvolte ústav DEF nebo jej zakomentujte / Department in English - for use of this entry select DEF above or comment it out
}

% Rozšířený abstrakt (cca 3 normostrany) - lze definovat zde nebo výše / Extended abstract (approximately 3 standard pages) - can be defined here or above
%\extendedabstract{Do tohoto odstavce bude zapsán výtah (abstrakt) práce v českém (slovenském) jazyce.}

% nastavení délky bloku s titulkem pro úpravu zalomení řádku - lze definovat zde nebo výše / setting the length of a block with a thesis title for adjusting a line break - can be defined here or above
%\titlelength{14.5cm}


% řeší první/poslední řádek odstavce na předchozí/následující stránce
% solves first/last row of the paragraph on the previous/next page
\clubpenalty=10000
\widowpenalty=10000

\usepackage{textcomp} % R copyright

\newcommand{\zl}[2]{\dfrac{\text{#1}}{\text{#2}}}
\newcommand{\vz}[3]{\text{#1} = \zl{#2}{#3}}

\begin{document}
  % Vysazeni titulnich stran / Typesetting of the title pages
  % ----------------------------------------------
  \maketitle
  % Obsah
  % ----------------------------------------------
  \setlength{\parskip}{0pt}

  {\hypersetup{hidelinks}\tableofcontents}
  
  % Seznam obrazku a tabulek (pokud prace obsahuje velke mnozstvi obrazku, tak se to hodi)
  % List of figures and list of tables (if the thesis contains a lot of pictures, it is good)
  \ifczech
    \renewcommand\listfigurename{Seznam obrázků}
  \fi
  \ifslovak
    \renewcommand\listfigurename{Zoznam obrázkov}
  \fi
  % \listoffigures
  
  \ifczech
    \renewcommand\listtablename{Seznam tabulek}
  \fi
  \ifslovak
    \renewcommand\listtablename{Zoznam tabuliek}
  \fi
  % \listoftables 

  \ifODSAZ
    \setlength{\parskip}{0.5\bigskipamount}
  \else
    \setlength{\parskip}{0pt}
  \fi

  % vynechani stranky v oboustrannem rezimu
  % Skip the page in the two-sided mode
  \iftwoside
    \cleardoublepage
  \fi

  % Text prace / Thesis text
  % ----------------------------------------------
  \chapter{Úvod}
Tato práce popisuje 2 související věci?. 
1)Prvním tématem je provádění finanční analýzy obecně -- shrnutí cílů a motivace, popis používaných metod, vstupních a výstupních dat, výpočtu samotného a intepretaci výsledků a omezení při dedukci?. Druhým tématem je popis programu, který takovou analýzu realizuje a který bude v rámci této diplomové práce implementován.

Finanční analýza podniku je proces, při kterém jsou získaná data (...?) podrobena rozboru a s využitím různých metod jsou vypočtena nová data s novou vypovídající hodnotou, mimo jiné i o finančním zdraví podniku. Tato data se nazývají finanční ukazatele. Lze je využít k nalezení příčin současného stavu podniku, ..., a do jisté míry? i k predikci stavu podniku v budoucnu.
různý výstup analýzy - čísla, grafy, peníze, textová forma

Text práce je členěn do X kapitol. Jelikož je finanční analýza poměrně rozsáhlá oblast, celá příští kapitola je věnována teorii. Nejprve je nezbytné vysvětlit , jelikož se některé z nich budou v textu objevovat často a běžný čtenář? se s nimi nemusel dosud setkat. + fungování podniku, zákon , ...
Ve třetí kapitole bude popsán návrh zvolených metod vyhodnocování finančního zdraví podniku, jeho interpretace a také návrh programového řešení - architektura, implementační jazyk, formát vstupu a výstupu. Čtvrtá kapitola se zabývá realizací návrhu, implementačními detaily. V závěrečné kapitole jsou shrnuty a zhodnoceny dosažené výsledky, přínosy této práce a návrhy na její možná rozšíření.

Pro lepší demonstraci budou v této práci analyzovány výsledky hospodaření společnosti Krajči plus s.r.o., která se zabývá... Výsledky hospodaření lze najít ... .

%=========================================================================
\chapter{Problematika finanční analýzy}

vykaz zisku a ztraty + rozvaha - z uctove osnovy na konci roku
vynosy, naklady X prijmy, vydaje (to jsou uz primo penize)
hruby\_zisk=vynosy-naklady
rozvaha=majetek firmy (at vim, k cemu pribude zisk)

\section{Vstupní data}
Abychom mohli provést finanční analýzu, musíme mít data, ze kterých budeme vycházet. Množina vstupních dat není nijak omezena a může být vskutku rozsáhlá -- při výpočtu lze zohlednit nejrůznější faktory. Klíčová jsou však především data o vlastním hospodaření podniku, za jejichž evidenci je zodpovědná účetní jednotka podniku, která je v průběhu roku zapisuje do tzv účtové osnovy.
\section{Účtová osnova}
Účtová osnova je 
Větší? společnosti jsou ze zákona povinny dělat? každoročně účetní závěrku. Ta má v ČR dvě povinné části - výkaz zisku a ztráty a rozvaha. 
\section{Výkaz zisku a ztráty}
\section{Rozvaha}
Narozdíl od výkazu zisku a ztráty jsou při sestavování? rozvahy data? vázána k jednomu určitému okamžiku, takzvanému rozvahovému dni, který je posledním dnem účetního období, zpravidla tedy posledním dnem kalendářního roku.
-aktiva
-pasiva

%#########################################################################
\section{Absolutní ukazatele}

\subsection{Horizontální analýza}
Při horizontální analýze je určitá položka (řádek) uzávěrky? porovnávána se stejnou položkou za jiné časové období. Při výpočtu tedy porovnáváme výsledky jedné položky napříč různými obdobími - proto horizontální analýza. U jednotlivých položek zvlášť je vypočtena meziroční změna a její procentuální vyjádření. Tento postup je proti ostatním velmi? přímočarý?
-používá se především pro ...
-nezohledňuje inflaci, ...

\subsection{Vertikální analýza}
Přívlastek vertikální vychází tak jako u horizontální analýzy ze způsobu výpočtu z uzávěrky. Při vertikální analýze jsou počítány ukazatele vycházející z poměru jednotlivých položek (řádků) v rámci jednoho roku.
%#########################################################################

\section{Poměrové ukazatele}
Existuje několik různých ukazatelů. Různé ukazatele slouží různým účelům, různé společnosti se řídí různými ukazateli. Ukazatele se počítají z výsledovky a rozvahy. Většina ukazatelů je vyjádřena poměrem výstupu k nějaké základní jednotce, kterou je 1 Kč položky? ve jmenovateli.

\subsection{Ukazatelé rentability}

Ukazatelé rentability (výnosnosti, ziskovosti) patří v praxi mezi nejsledovanější ukazatele. Vyjadřují míru výnosnosti hospodaření podniku, nikoliv však z pohledu vlastníka. pomocí poměru  jaké bylo dosaženo na základě vloženého kapitálu. Obecný tvar výpočtu je tedy $$\zl{výnos}{prostředky}$$

\begin{itemize}
\item{ROA -- rentabilita aktiv}

return on assets
ROA = zisková marže $*$ obrat celkových aktiv = $\zl{čistý zisk}{tržby} * \zl{tržby}{celková aktiva}$

\item{ROE -- rentabilita vlastního kapitálu}

return on equity
Vyjadřuje míru zhodnocení kapitálu -- kolik korun čistého zisku připadá na 1 korunu vlastního kapitálu. Pro akcionáře je tedy jedním z klíčových ukazatelů.

\item{ROCE -- rentabilita investovaného kapitálu}

return on capital employed

\item{ROS -- rentabilita tržeb}

return on sales

\item{ROC -- rentabilita nákladů}

return on costs
\end{itemize}


\subsection{Ukazatelé likvidity}
Pojem likvidita bývá nesprávně zaměňován s úzce souvisejícími pojmy likvidnost a solventnost. Likvidnost je vlastnost jednotlivých složek aktiv podniku vyjadřující schopnost přeměny těchto složek v peněžní prostředky v co nejkratším čase a s minimálními ztrátami\cite{uk_likv}. Přeměnou můžeme rozumět například prodej zásob nebo inkasování pohledávek. 

Solventnost neboli platební schopnost je definována jako schopnost subjektu, v našem případě podniku, včas splácet své finanční závazky.

Likvidita podniku dává tyto pojmy do souvislosti -- vyjadřuje schopnost přeměnit aktiva podniku na peněžní prostředky (využít jejich likvidnosti) za účelem včasného uhrazení všech splatných závazků (a tím pádem být solventní).

Ukazatelé likvidity potom udávají poměr toho, čím je možné platit to co se musi zaplatit

\begin{itemize}
\item{Běžná likvidita} 

CR - Current Ratio

Poměr ukazuje, kolikrát pokrývají oběžná aktiva krátkodobé závazky, jinými slovy, kolikrát je podnik schopen splnit své krátkodobé závazky z peněžních prostředků, které by získal přeměnou z oběžných aktiv.

Výpočet zahrnuje i zásoby, jejichž konkrétní ocenění ve velké míře ovlivňuje výsledek.
Ukazatel se tak dá považovat za měřítko budoucí solventnosti podniku, optimální hodnota je CR $= 1.5$
$$\zl{oběžná aktiva}{krátkodobé závazky}$$

\item{Pohotová likvidita} 

stejný výpočet, pouze bez zásob. U podniku z oblasti služeb se blíží běžné likviditě, u výrobních podniků se naopak spíše liší.
QR - Quick Ratio
$$\zl{oběžná aktiva $-$ zásoby}{krátkodobé závazky}$$

\item{Peněžní (okamžitá) likvidita}

CR - Cash Ratio

$$\zl{peněžní prostředky $+$ ekvivalenty}{krátkodobé závazky}$$
pěněžními prostředky se myslí peníze v hotovosti a na běžných účtech, ekvivalenty mohou být například šeky nebo obchdovatelné cenné papíry. Pro podnik je žádoucí, aby platilo CR > 0,2, čímž je zajištěna likvidita.
\end{itemize}


\subsection{Ukazatelé aktivity}
\begin{itemize}
\item{Obrat celkových aktiv}
\item{Obrat zásob}
\item{Obrat pohledávek}
\item{Obrat závazků}
\end{itemize}

\subsection{Ukazatelé produktivity}

\subsection{Ukazatelé zadluženosti}
Zadluženost podniku je vyjádřena poměrem vlastních a cizích zdrojů financování podniku. Cizí zdroje pro podnik zpravidla znamenají dluh, závazek. Zadluženost obecně nemusí být negativní charakteristikou -- využití cizích zdrojů vede k zesílení tzv. finančního pákového efektu?vysvetl, který pozitivně přispívá k rentabilitě vlastního kapitálu\cite{uk_zadl}.
\begin{itemize}
\item{Úrokové krytí}
\end{itemize}


\subsection{IRR -- vnitřní výnosové procento}
internal rate of return

\subsection{EVA}
economic value added



\section{Interpretace výsledků}
Vzorce a postupy pro výpočet jednotlivých ukazatelů jsou jasně (matematicky) dané. Ukazatel je však jen číslo a problémem tak může být jeho interpretace -- v některých případech nelze jednoznačně určit příčinu, důsledek nebo obecně význam jeho hodnoty. 

\section{Existující software}
\subsection{Equanta\sffamily\textregistered}

%=========================================================================
\chapter{Návrh řešení}
\section{Vybrané metody vyhodnocování finančního zdraví}
Ukazatele jsou ve skutečnosti jen čísla, kterým rozumí spíše ekonomové - běžný uživatel by z nich nejspíš nic nevydedukoval. Proto je žádoucí, aby byly výstupem nejen tato čísla, ale i nějaký srozumitelný závěr?.
Porovnání s minulým rokem
Porovnání s ukazateli v odvětví

Horizontální analýza
- sloupcový nebo spojnicový graf
- vyber polozky vysledovky (checkbox/selectbox ...?)
- checkboxy pro roky

- co nejvice ukazatelu - komplexni

\section{Návrh programového řešení}
Program bude mít webové rozhraní, což má samozřejmě své výhody - uživatel může k programu přistupovat odkudkoliv (a to z různých platforem), používání programu není podmíněno jeho instalací a z pohledu vývojáře je mnohem jednodušší jeho údržba (vydávání nových verzí).

- defaultni nastaveni, ale vysoka uroven prizbusobeni, vse svazano s uctem
 nekolik ruznych podniku na uziv(financni analytik)
 nekolik uzaverek na podnik (ruzne roky)
 zobrazene/preferovane ukazatele
 
- vstup - moznost vyplnit rucne, krome pdf na MPO? se odevzdava ZZ a R v xml na financak
- vstup - parametrizace parseru

Uživateli bude pro snadnější práci umožněno vytoření vlastního účtu. Systém tak bude schopen personalizovat obsah dle nastavených předvoleb. S účtem budou svázana i data za jednotlivá časová obdoví v minulosti, která lze nahrát a uchovat pro další použití, například porovnání nebo zpřesnění analýzy v budoucnu.

-React(Redux,Navigation,bootstrap)
-Firebase?

-navrh testů (rozvaha,ZZ,vypočítané uk)

%=========================================================================
\chapter{Implementace a testování}
%=========================================================================
\chapter{Závěr}

  
  % Kompilace po částech (viz výše, nutno odkomentovat)
  % Compilation piecewise (see above, it is necessary to uncomment it)
  %\subfile{projekt-01-uvod-introduction}
  % ...
  %\subfile{chapters/projekt-05-conclusion}


  % Pouzita literatura / Bibliography
  % ----------------------------------------------
\ifslovak
  \makeatletter
  \def\@openbib@code{\addcontentsline{toc}{chapter}{Literatúra}}
  \makeatother
  \bibliographystyle{bib-styles/czechiso}
\else
  \ifczech
    \makeatletter
    \def\@openbib@code{\addcontentsline{toc}{chapter}{Literatura}}
    \makeatother
    \bibliographystyle{bib-styles/czechiso}
  \else 
    \makeatletter
    \def\@openbib@code{\addcontentsline{toc}{chapter}{Bibliography}}
    \makeatother
    \bibliographystyle{bib-styles/englishiso}
  %  \bibliographystyle{alpha}
  \fi
\fi
  \begin{flushleft}
  \bibliography{projekt-20-literatura-bibliography}
  \end{flushleft}

  % vynechani stranky v oboustrannem rezimu
  % Skip the page in the two-sided mode
  \iftwoside
    \cleardoublepage
  \fi

  % Prilohy / Appendices
  % ---------------------------------------------
  \appendix
\ifczech
  \renewcommand{\appendixpagename}{Přílohy}
  \renewcommand{\appendixtocname}{Přílohy}
  \renewcommand{\appendixname}{Příloha}
\fi
\ifslovak
  \renewcommand{\appendixpagename}{Prílohy}
  \renewcommand{\appendixtocname}{Prílohy}
  \renewcommand{\appendixname}{Príloha}
\fi
%  \appendixpage

% vynechani stranky v oboustrannem rezimu
% Skip the page in the two-sided mode
%\iftwoside
%  \cleardoublepage
%\fi
  
\ifslovak
%  \section*{Zoznam príloh}
%  \addcontentsline{toc}{section}{Zoznam príloh}
\else
  \ifczech
%    \section*{Seznam příloh}
%    \addcontentsline{toc}{section}{Seznam příloh}
  \else
%    \section*{List of Appendices}
%    \addcontentsline{toc}{section}{List of Appendices}
  \fi
\fi
  \startcontents[chapters]
  \setlength{\parskip}{0pt}
  % seznam příloh / list of appendices
  % \printcontents[chapters]{l}{0}{\setcounter{tocdepth}{2}}
  
  \ifODSAZ
    \setlength{\parskip}{0.5\bigskipamount}
  \else
    \setlength{\parskip}{0pt}
  \fi
  
  % vynechani stranky v oboustrannem rezimu
  \iftwoside
    \cleardoublepage
  \fi
  
  % Přílohy / Appendices
  % Tento soubor nahraďte vlastním souborem s přílohami (nadpisy níže jsou pouze pro příklad)
% This file should be replaced with your file with an appendices (headings below are examples only)

% Umístění obsahu paměťového média do příloh je vhodné konzultovat s vedoucím
% Placing of table of contents of the memory media here should be consulted with a supervisor
%\chapter{Obsah přiloženého paměťového média}

%\chapter{Manuál}

%\chapter{Konfigurační soubor} % Configuration file

%\chapter{RelaxNG Schéma konfiguračního souboru} % Scheme of RelaxNG configuration file

%\chapter{Plakát} % poster

\chapter{Jak pracovat s touto šablonou}
\label{jak}

V této kapitole je uveden popis jednotlivých částí šablony, po kterém následuje stručný návod, jak s touto šablonou pracovat. 

Jedná se o přechodnou verzi šablony. Nová verze bude zveřejněna do~konce roku 2017 a~bude navíc obsahovat nové pokyny ke správnému využití šablony, závazné pokyny k~vypracování bakalářských a diplomových prací (rekapitulace pokynů, které jsou dostupné na~webu) a nezávazná doporučení od vybraných vedoucích, která již teď najdete na~webu (viz odkazy v souboru s literaturou). Jediné soubory, které se v nové verzi změní, budou \texttt{projekt-01-kapitoly-chapters.tex} a \texttt{projekt-30-prilohy-appendices.tex}, jejichž obsah každý student vymaže a nahradí vlastním. Šablonu lze tedy bez problémů využít i~v~současné verzi.

\section*{Popis částí šablony}

Po rozbalení šablony naleznete následující soubory a adresáře:
\begin{DESCRIPTION}
  \item [bib-styles] Styly literatury (viz níže). 
  \item [obrazky-figures] Adresář pro Vaše obrázky. Nyní obsahuje placeholder.pdf (tzv. TODO obrázek, který lze použít jako pomůcku při tvorbě technické zprávy), který se s prací neodevzdává. Název adresáře je vhodné zkrátit, aby byl jen ve zvoleném jazyce.
  \item [template-fig] Obrázky šablony (znak VUT).
  \item [fitthesis.cls] Šablona (definice vzhledu).
  \item [Makefile] Makefile pro překlad, počítání normostran, sbalení apod. (viz níže).
  \item [projekt-01-kapitoly-chapters.tex] Soubor pro Váš text (obsah nahraďte).
  \item [projekt-20-literatura-bibliography.bib] Seznam literatury (viz níže).
  \item [projekt-30-prilohy-appendices.tex] Soubor pro přílohy (obsah nahraďte).
  \item [projekt.tex] Hlavní soubor práce -- definice formálních částí.
\end{DESCRIPTION}

Výchozí styl literatury (czechiso) je od Ing. Martínka, přičemž anglická verze (englishiso) je jeho překladem s drobnými modifikacemi. Oproti normě jsou v něm určité odlišnosti, ale na FIT je dlouhodobě akceptován. Alternativně můžete využít styl od Ing. Radima Loskota nebo od Ing. Radka Pyšného\footnote{BP Ing. Radka Pyšného \url{http://www.fit.vutbr.cz/study/DP/BP.php?id=7848}}. Alternativní styly obsahují určitá vylepšení, ale zatím nebyly řádně otestovány větším množstvím uživatelů. Lze je považovat za beta verze pro zájemce, kteří svoji práci chtějí mít dokonalou do detailů a neváhají si nastudovat detaily správného formátování citací, aby si mohli ověřit, že je vysázený výsledek v pořádku.

Makefile kromě překladu do PDF nabízí i další funkce:
\begin{itemize}
  \item přejmenování souborů (viz níže),
  \item počítání normostran,
  \item spuštění vlny pro doplnění nezlomitelných mezer,
  \item sbalení výsledku pro odeslání vedoucímu ke kontrole (zkontrolujte, zda sbalí všechny Vámi přidané soubory, a případně doplňte).
\end{itemize}

Nezapomeňte, že vlna neřeší všechny nezlomitelné mezery. Vždy je třeba manuální kontrola, zda na konci řádku nezůstalo něco nevhodného -- viz Internetová jazyková příručka\footnote{Internetová jazyková příručka \url{http://prirucka.ujc.cas.cz/?id=880}}.

\paragraph {Pozor na číslování stránek!} Pokud má obsah 2 strany a na 2. jsou jen \uv{Přílohy} a~\uv{Seznam příloh} (ale žádná příloha tam není), z nějakého důvodu se posune číslování stránek o 1 (obsah \uv{nesedí}). Stejný efekt má, když je na 2. či 3. stránce obsahu jen \uv{Literatura} a~je možné, že tohoto problému lze dosáhnout i jinak. Řešení je několik (od~úpravy obsahu, přes nastavení počítadla až po sofistikovanější metody). \textbf{Před odevzdáním proto vždy překontrolujte číslování stran!}


\section*{Doporučený postup práce se šablonou}

\begin{enumerate}
  \item \textbf{Zkontrolujte, zda máte aktuální verzi šablony.} Máte-li šablonu z předchozího roku, na stránkách fakulty již může být novější verze šablony s~aktualizovanými informacemi, opravenými chybami apod.
  \item \textbf{Zvolte si jazyk}, ve kterém budete psát svoji technickou zprávu (česky, slovensky nebo anglicky) a svoji volbu konzultujte s vedoucím práce (nebyla-li dohodnuta předem). Pokud Vámi zvoleným jazykem technické zprávy není čeština, nastavte příslušný parametr šablony v souboru projekt.tex (např.: \verb|documentclass[english]{fitthesis}| a přeložte prohlášení a poděkování do~angličtiny či slovenštiny.
  \item \textbf{Přejmenujte soubory.} Po rozbalení je v šabloně soubor \texttt{projekt.tex}. Pokud jej přeložíte, vznikne PDF s technickou zprávou pojmenované \texttt{projekt.pdf}. Když vedoucímu více studentů pošle \texttt{projekt.pdf} ke kontrole, musí je pracně přejmenovávat. Proto je vždy vhodné tento soubor přejmenovat tak, aby obsahoval Váš login a (případně zkrácené) téma práce. Vyhněte se však použití mezer, diakritiky a speciálních znaků. Vhodný název může být např.: \uv{\texttt{xlogin00-Cisteni-a-extrakce-textu.tex}}. K přejmenování můžete využít i přiložený Makefile:
\begin{verbatim}
make rename NAME=xlogin00-Cisteni-a-extrakce-textu
\end{verbatim}
  \item Vyplňte požadované položky v souboru, který byl původně pojmenován \texttt{projekt.tex}, tedy typ, rok (odevzdání), název práce, svoje jméno, ústav (dle zadání), tituly a~jméno vedoucího, abstrakt, klíčová slova a další formální náležitosti.
  \item Nahraďte obsah souborů s kapitolami práce, literaturou a přílohami obsahem svojí technické zprávy. Jednotlivé přílohy či kapitoly práce může být výhodné uložit do~samostatných souborů -- rozhodnete-li se pro toto řešení, je doporučeno zachovat konvenci pro názvy souborů, přičemž za číslem bude následovat název kapitoly. 
  \item Nepotřebujete-li přílohy, zakomentujte příslušnou část v \texttt{projekt.tex} a příslušný soubor vyprázdněte či smažte. Nesnažte se prosím vymyslet nějakou neúčelnou přílohu jen proto, aby daný soubor bylo čím naplnit. Vhodnou přílohou může být obsah přiloženého paměťového média.
  \item Nascanované zadání uložte do souboru \texttt{zadani.pdf} a povolte jeho vložení do práce parametrem šablony v projekt.tex (\verb|documentclass[zadani]{fitthesis}|).
  \item Nechcete-li odkazy tisknout barevně (tedy červený obsah -- bez konzultace s vedoucím nedoporučuji), budete pro tisk vytvářet druhé PDF s tím, že nastavíte parametr šablony pro tisk: (\verb|documentclass[zadani,print]{fitthesis}|).  Barevné logo se nesmí tisknout černobíle!
  \item Vzor desek, do kterých bude práce vyvázána, si vygenerujte v informačním systému fakulty u zadání. Pro disertační práci lze zapnout parametrem v šabloně (více naleznete v souboru fitthesis.cls).
  \item Nezapomeňte, že zdrojové soubory i (obě verze) PDF musíte odevzdat na CD či jiném médiu přiloženém k technické zprávě.
\end{enumerate}

Obsah práce se generuje standardním příkazem \tt \textbackslash tableofcontents \rm (zahrnut v šabloně). Přílohy jsou v něm uvedeny úmyslně.

\subsection*{Pokyny pro oboustranný tisk}
\begin{itemize}
\item \textbf{Oboustranný tisk je doporučeno konzultovat s vedoucím práce.}
\item Je-li práce tištěna oboustranně a její tloušťka je menší než tloušťka desek, nevypadá to dobře.
\item Zapíná se parametrem šablony: \verb|\documentclass[twoside]{fitthesis}|
\item Po vytištění oboustranného listu zkontrolujte, zda je při prosvícení sazební obrazec na obou stranách na stejné pozici. Méně kvalitní tiskárny s duplexní jednotkou mají často posun o 1--3 mm. Toto může být u některých tiskáren řešitelné tak, že vytisknete nejprve liché stránky, pak je dáte do stejného zásobníku a vytisknete sudé.
\item Za titulním listem, obsahem, literaturou, úvodním listem příloh, seznamem příloh a případnými dalšími seznamy je třeba nechat volnou stránku, aby následující část začínala na liché stránce (\textbackslash cleardoublepage).
\item  Konečný výsledek je nutné pečlivě překontrolovat.
\end{itemize}

\subsection*{Styl odstavců}

Odstavce se zarovnávají do bloku a pro jejich formátování existuje více metod. U papírové literatury je častá metoda s~použitím odstavcové zarážky, kdy se u~jednotlivých odstavců textu odsazuje první řádek odstavce asi o~jeden až dva čtverčíky (vždy o~stejnou, předem zvolenou hodnotu), tedy přibližně o~dvě šířky velkého písmene M základního textu. Poslední řádek předchozího odstavce a~první řádek následujícího odstavce se v~takovém případě neoddělují svislou mezerou. Proklad mezi těmito řádky je stejný jako proklad mezi řádky uvnitř odstavce. \cite{fitWeb} Další metodou je odsazení odstavců, které je časté u elektronické sazby textů. První řádek odstavce se při této metodě neodsazuje a mezi odstavce se vkládá vertikální mezera o~velikosti 1/2 řádku. Obě metody lze v kvalifikační práci použít, nicméně často je vhodnější druhá z uvedených metod. Metody není vhodné kombinovat.

Jeden z výše uvedených způsobů je v šabloně nastaven jako výchozí, druhý můžete zvolit parametrem šablony \uv{\tt odsaz\rm }.

\subsection*{Užitečné nástroje}
\label{nastroje}

Následující seznam není výčtem všech využitelných nástrojů. Máte-li vyzkoušený osvědčený nástroj, neváhejte jej využít. Pokud však nevíte, který nástroj si zvolit, můžete zvážit některý z následujících:

\begin{description}
	\item[\href{http://miktex.org/download}{MikTeX}] \LaTeX{} pro Windows -- distribuce s jednoduchou instalací a vynikající automatizací stahování balíčků.
	\item[\href{http://texstudio.sourceforge.net/}{TeXstudio}] Přenositelné opensource GUI pro \LaTeX{}.  Ctrl+klik umožňuje přepínat mezi zdrojovým textem a PDF. Má integrovanou kontrolu pravopisu, zvýraznění syntaxe apod. Pro jeho využití je nejprve potřeba nainstalovat MikTeX.
	\item[\href{http://www.winedt.com/}{WinEdt}] Ve Windows je dobrá kombinace WinEdt + MiKTeX. WinEdt je GUI pro Windows, pro jehož využití je nejprve potřeba nainstalovat \href{http://miktex.org/download}{MikTeX} či \href{http://www.tug.org/texlive/}{TeX Live}. 
	\item[\href{http://kile.sourceforge.net/}{Kile}] Editor pro desktopové prostředí KDE (Linux). Umožňuje živé zobrazení náhledu. Pro jeho využití je potřeba mít nainstalovaný \href{http://www.tug.org/texlive/}{TeX Live} a Okular. 
	\item[\href{http://jabref.sourceforge.net/download.php}{JabRef}] Pěkný a jednoduchý program v Javě pro správu souborů s bibliografií (literaturou). Není potřeba se nic učit -- poskytuje jednoduché okno a formulář pro editaci položek.
	\item[\href{https://inkscape.org/en/download/}{InkScape}] Přenositelný opensource editor vektorové grafiky (SVG i PDF). Vynikající nástroj pro tvorbu obrázků do odborného textu. Jeho ovládnutí je obtížnější, ale výsledky stojí za to.
	\item[\href{https://git-scm.com/}{GIT}] Vynikající pro týmovou spolupráci na projektech, ale může výrazně pomoci i jednomu autorovi. Umožňuje jednoduché verzování, zálohování a přenášení mezi více počítači.
	\item[\href{http://www.overleaf.com/}{Overleaf}] Online nástroj pro \LaTeX{}. Přímo zobrazuje náhled a umožňuje jednoduchou spolupráci (vedoucí může průběžně sledovat psaní práce), vyhledávání ve zdrojovém textu kliknutím do PDF, kontrolu pravopisu apod. Zdarma jej však lze využít pouze s určitými omezeními (někomu stačí na disertaci, jiný na ně může narazit i při psaní bakalářské práce) a pro dlouhé texty je pomalejší.
\end{description}

Pozn.: Overleaf nepoužívá Makefile v šabloně -- aby překlad fungoval, je nutné kliknout pravým tlačítkem na \tt projekt.tex \rm a zvolit \uv{Set as Main File}.


\subsection*{Užitečné balíčky pro \LaTeX}

Studenti při sazbě textu často řeší stejné problémy. Některé z nich lze vyřešit následujícími balíčky pro \LaTeX:

\begin{itemize}
  \item \verb|amsmath| -- rozšířené možnosti sazby rovnic,
  \item \verb|float, afterpage, placeins| -- úprava umístění obrázků,
  \item \verb|fancyvrb, alltt| -- úpravy vlastností prostředí Verbatim, 
  \item \verb|makecell| -- rozšíření možností tabulek,
  \item \verb|pdflscape, rotating| -- natočení stránky o 90 stupňů (pro obrázek či tabulku),
  \item \verb|hyphenat| -- úpravy dělení slov,
  \item \verb|picture, epic, eepic| -- přímé kreslení obrázků.
\end{itemize}

Některé balíčky jsou využity přímo v šabloně (v dolní části souboru fitthesis.cls). Nahlédnutí do jejich dokumentace může být rovněž užitečné.

Sloupec tabulky zarovnaný vlevo s pevnou šířkou je v šabloně definovaný \uv{L} (používá se jako \uv{p}).


  
  % Kompilace po částech (viz výše, nutno odkomentovat)
  % Compilation piecewise (see above, it is necessary to uncomment it)
  %\subfile{projekt-30-prilohy-appendices}
  
\end{document}
