\chapter{Úvod}
Tato práce popisuje 2 související věci?. 
1)Prvním tématem je provádění finanční analýzy obecně -- shrnutí motivace a cílů, popis pojmů, používaných metod a strategií, vstupních a výstupních dat, výpočtu samotného a intepretaci výsledků. Druhým tématickým celkem je popis implementace a fungování software, který finanční analýzu realizuje a který bude v rámci této diplomové práce implementován.

Finanční analýza podniku je proces, při kterém jsou získaná data (...?) podrobena rozboru a s využitím různých metod jsou vypočtena nová data s novou vypovídající hodnotou, mimo jiné i o finančním zdraví podniku. Tato data se nazývají finanční ukazatele. Lze je využít k nalezení příčin současného stavu podniku, ..., a do jisté míry? i k predikci stavu podniku v budoucnu.
různý výstup analýzy - čísla, grafy, peníze, textová forma

Text práce je členěn do pěti kapitol. Jelikož je finanční analýza poměrně rozsáhlá oblast, celá příští kapitola je věnována teorii. Nejprve je nezbytné vysvětlit , jelikož se některé z nich budou v textu objevovat často a běžný čtenář? se s nimi nemusel dosud setkat. + fungování podniku, zákon , ...
Ve třetí kapitole bude popsán návrh zvolených metod vyhodnocování finančního zdraví podniku, jeho interpretace a také návrh programového řešení - architektura, implementační jazyk, formát vstupu a výstupu. Čtvrtá kapitola se zabývá realizací návrhu, implementačními detaily. V závěrečné kapitole jsou shrnuty a zhodnoceny dosažené výsledky, přínosy této práce a návrhy na její možná rozšíření.

Pro lepší demonstraci budou v této práci analyzovány výsledky hospodaření společnosti ?, která se zabývá ?. Výsledky hospodaření lze najít ?.

%=========================================================================
\chapter{Problematika finanční analýzy}

vykaz zisku a ztraty + rozvaha - z uctove osnovy na konci roku
vynosy, naklady X prijmy, vydaje (to jsou uz primo penize)
hruby\_zisk=vynosy-naklady
rozvaha=majetek firmy (at vim, k cemu pribude zisk)

\section{Vstupní data}
Abychom mohli provést finanční analýzu, musíme mít data, ze kterých budeme vycházet. Množina vstupních dat není nijak omezena a může být vskutku rozsáhlá -- při výpočtu lze zohlednit nejrůznější faktory. Klíčová jsou však především data o vlastním hospodaření podniku, za jejichž evidenci je zodpovědná účetní jednotka podniku, která je v průběhu roku zapisuje do tzv účtové osnovy.
\section{Účtová osnova}
Účtová osnova je 
Větší? společnosti jsou ze zákona povinny dělat? každoročně účetní závěrku. Ta má v ČR dvě povinné části - výkaz zisku a ztráty a rozvaha. 
\section{Výkaz zisku a ztráty}
\section{Rozvaha}
Narozdíl od výkazu zisku a ztráty jsou při sestavování? rozvahy data? vázána k jednomu určitému okamžiku, takzvanému rozvahovému dni, který je posledním dnem účetního období, zpravidla tedy posledním dnem kalendářního roku.
-aktiva
-pasiva

%#########################################################################
\section{Absolutní ukazatele}

\subsection{Horizontální analýza}
Při horizontální analýze je určitá položka (řádek) uzávěrky? porovnávána se stejnou položkou za jiné časové období. Při výpočtu tedy porovnáváme výsledky jedné položky napříč různými obdobími - proto horizontální analýza. U jednotlivých položek zvlášť je vypočtena meziroční změna a její procentuální vyjádření. Tento postup je proti ostatním velmi? přímočarý?
-používá se především pro ...
-nezohledňuje inflaci, ...

\subsection{Vertikální analýza}
Přívlastek vertikální vychází tak jako u horizontální analýzy ze způsobu výpočtu z uzávěrky. Při vertikální analýze jsou počítány ukazatele vycházející z poměru jednotlivých položek (řádků) v rámci jednoho roku.
%#########################################################################

\section{Poměrové ukazatele}
Existuje několik různých ukazatelů. Různé ukazatele slouží různým účelům, různé společnosti se řídí různými ukazateli. Ukazatele se počítají z výsledovky a rozvahy. Většina ukazatelů je vyjádřena poměrem výstupu k nějaké základní jednotce, kterou je 1 Kč položky? ve jmenovateli.

\subsection{Ukazatelé rentability}

Ukazatelé rentability (výnosnosti, ziskovosti, návratnosti) patří v praxi mezi nejsledovanější ukazatele. Vyjadřují míru výnosnosti hospodaření podniku, spíše však z pohledu podniku samotného, nežli vlastníka. Obecný tvar výpočtu je tedy 
$$\zl{výnos}{prostředky}$$

\subsubsection{ROA -- rentabilita aktiv}

return on assets

ROA dává do poměru zisk s celkovými aktivy podniku, bez ohledu na to, zda jsou aktiva vlastní nebo cizí, dlouhodobá nebo krátkodobá. Tím vyjadřuje, jak efektivně dokáže podnik naložit se svým majetkem?. V čitateli se může objevit zisk před i po zdanění respektive zúročení.

$$\text{ROA} = \zl{EBIT}{A}  \text{ROA} = \zl{EBIT $*(1-t)$}{A}$$ 
$$\text{ROA} = \zl{EAT}{A}  \text{ROA} = \zl{EAT $+$ úroky $*(1-t)$}{A}$$

\subsubsection{ROE -- rentabilita vlastního kapitálu}

return on equity
Při výpočtu ROE je ve jmenovateli místo všech aktiv hodnota vlastního kapitálu. Tím vyjadřuje míru jeho zhodnocení -- kolik korun (čistého) zisku připadá na 1 korunu vlastního kapitálu. Pro vlastníky je tedy jedním z klíčových ukazatelů. Ve srovnání s ROA

$$\text{ROE} = \zl{čistý zisk}{vlastní kapitál}$$ 

\subsubsection{ROCE -- rentabilita investovaného kapitálu}

return on capital employed

$$\text{ROCE} = \zl{EBIT}{vl. kapitál + rezervy + dlouhodobé závazky + bankovní úvěry dlouhodobé}$$ 

\subsubsection{ROS -- rentabilita tržeb}

return on sales

$$\text{ROS} = \zl{EBIT}{tržby} \text{ROS} = \zl{EAT}{tržby}$$ 

\subsubsection{ROC -- rentabilita nákladů}

return on costs


\subsection{Ukazatelé likvidity}
Pojem likvidita bývá nesprávně zaměňován s úzce souvisejícími pojmy likvidnost a solventnost. Likvidnost je vlastnost jednotlivých složek aktiv podniku vyjadřující schopnost přeměny těchto složek v peněžní prostředky v co nejkratším čase a s minimálními ztrátami\cite{uk_likv}. Přeměnou můžeme rozumět například prodej zásob nebo inkasování pohledávek. 

Solventnost neboli platební schopnost je definována jako schopnost subjektu, v našem případě podniku, včas splácet své finanční závazky.

Likvidita podniku dává tyto pojmy do souvislosti -- ukazatelé likvidity vyjadřují schopnost přeměnit vybraná aktiva podniku na peněžní prostředky (využít jejich likvidnosti) za účelem včasného uhrazení všech splatných závazků (a tím pádem být solventní).

Všechny ukazatele se počítají jako poměr toho, čím je možné platit, k tomu, co se musí zaplatit.

\subsubsection{Běžná likvidita} 

CR - Current Ratio

Poměr ukazuje, kolikrát pokrývají oběžná aktiva krátkodobé závazky, jinými slovy, kolikrát je podnik schopen uhradit své krátkodobé závazky z peněžních prostředků, které by získal přeměnou z oběžných aktiv.

Výpočet zahrnuje i zásoby, jejichž konkrétní ocenění ve velké míře ovlivňuje výsledek.
Ukazatel se tak dá považovat za měřítko budoucí solventnosti podniku, Uváděná optimální hodnota se v literaturách různí, zřejmě se bude lišit i v závisloti na typu podniku. Hodnota ukazatele by určitě neměla být menší než 1, což by znamenalo nutnost financovat závazky z dlouhodobých zdrojů financování a jinými nevhodnými způsoby. Také berme v potaz, že po pokrytí závazků by měly zbýt prostředky pro další činnost podniku.
Optimální hodnota uvedená v preferovaném zdroji je CR $\geq 1.5$. 

$$\zl{oběžná aktiva}{krátkodobé závazky}$$

\subsubsection{Pohotová likvidita} 

QR - Quick Ratio

stejný výpočet, pouze bez zásob. U podniku z oblasti služeb se blíží běžné likviditě, u výrobních podniků se naopak spíše liší.

$$\zl{oběžná aktiva $-$ zásoby}{krátkodobé závazky}$$

\subsubsection{Peněžní (okamžitá) likvidita}

CR - Cash Ratio

pěněžními prostředky se myslí peníze v hotovosti a na běžných účtech, ekvivalenty mohou být například šeky nebo obchdovatelné cenné papíry. Pro podnik je žádoucí, aby platilo CR > 0,2, čímž je zajištěna likvidita.

$$\zl{peněžní prostředky $+$ ekvivalenty}{krátkodobé závazky}$$


\subsection{Ukazatele aktivity}
Patří k tzv. mezivýkazových ukazatelům -- vstupními daty jsou položky jak z výkazu zisku a ztrát tak z rozvahy. Ukazatele aktivity vyjadřují, jak efektivně podnik nakládá se svými aktivy. Pokud jich vlastní více, než je nutné, vznikají zbytečné náklady a dochází ke snižení zisku. Jestliže jich má nedostatek, přichází o výnosy z potenciálních zakázek, o které se kvůli nedostatku aktiv nemůže ucházet.

Výsledkem je hodnota udávající rychlost nebo dobu obratu. Rychlost obratu vyjadřuje, kolikrát se určitá složka aktiv přemění za sledované období na peněžní prostředky. Doba obratu nám říká, jak dlouho trvá přeměna.

\subsubsection{Obrat celkových aktiv}

total assets turnover

Udává, kolikrát se obrátí celková aktiva za sledované časové období. Nevýhodou tohoto a některých dalších ukazatelů je povaha vstupních dat: tržby jsou zachyceny jejich sumou za celé období, kdežto hodnota aktiv se vztahuje ke konkrétnímu časovému okamžiku, kdy byla rozvaha vytvořena. Přesnějšího výsledku by se dalo dosáhnout použitím průměrné hodnoty celkových aktiv. 

$$\text{obrat celkových aktiv} = \zl{tržby}{celková aktiva}$$

\subsubsection{Obrat stálých aktiv}

fixed assets turnover

$$\text{obrat stálých aktiv} = \zl{tržby}{dlouhodobý hmotný majetek}$$

\subsubsection{Obrat zásob}

inventory turnover

$$\text{obrat zásob} = \zl{tržby}{zásoby}$$


\subsubsection{Obrat pohledávek}
\subsubsection{Obrat závazků}


\subsection{Ukazatele zadluženosti}
Zadluženost podniku je vyjádřena poměrem vlastních a cizích zdrojů financování podniku. Cizí zdroje pro podnik zpravidla znamenají dluh, závazek. Zadluženost obecně nemusí být negativní charakteristikou -- využití cizích zdrojů vede k zesílení tzv. finančního pákového efektu\footnote{finanční pákový efekt: bla bla bla?}, který pozitivně přispívá k rentabilitě vlastního kapitálu\cite{uk_zadl}.
\subsubsection{Úrokové krytí}

\subsection{Ukazatele produktivity}

\subsection{Další ukazatele}

\subsubsection{IRR -- vnitřní výnosové procento}
internal rate of return

\subsubsection{EVA}
economic value added



\section{Interpretace výsledků}
Vzorce a postupy pro výpočet jednotlivých ukazatelů jsou jasně (matematicky) dané. Ukazatel je však jen číslo a problémem tak může být jeho interpretace -- v některých případech nelze jednoznačně určit příčinu, důsledek nebo obecně význam jeho hodnoty. 

\section{Existující software}
\subsection{Equanta\sffamily\textregistered}

%=========================================================================
\chapter{Návrh řešení}
\section{Vybrané metody vyhodnocování finančního zdraví}
Ukazatele jsou ve skutečnosti jen čísla, kterým rozumí spíše ekonomové - běžný uživatel by z nich nejspíš nic nevydedukoval. Proto je žádoucí, aby byly výstupem nejen tato čísla, ale i nějaký srozumitelný závěr?.
Porovnání s minulým rokem
Porovnání s ukazateli v odvětví

Horizontální analýza
- sloupcový nebo spojnicový graf
- vyber polozky vysledovky (checkbox/selectbox ...?)
- checkboxy pro roky

- co nejvice ukazatelu - komplexni

\section{Návrh programového řešení}
Program bude mít webové rozhraní, což má samozřejmě své výhody - uživatel může k programu přistupovat odkudkoliv (a to z různých platforem), používání programu není podmíněno jeho instalací a z pohledu vývojáře je mnohem jednodušší jeho údržba (vydávání nových verzí).

- defaultni nastaveni, ale vysoka uroven prizbusobeni, vse svazano s uctem
 nekolik ruznych podniku na uziv(financni analytik)
 nekolik uzaverek na podnik (ruzne roky)
 zobrazene/preferovane ukazatele
 
VSTUP
- moznost vyplnit rucne, krome pdf na MPO? se odevzdava ZZ a R v xml na financak (k danovemu priz)
- vstup - parametrizace parseru


VYPOCET
- nastaveni optimalnich hodnot (rozmezi) pro ukazatele likvidity
- vypocet obrat celkovych aktiv - prumerna hodnota celkovych aktiv

Uživateli bude pro snadnější práci umožněno vytoření vlastního účtu. Systém tak bude schopen personalizovat obsah dle nastavených předvoleb. S účtem budou svázana i data za jednotlivá časová obdoví v minulosti, která lze nahrát a uchovat pro další použití, například porovnání nebo zpřesnění analýzy v budoucnu.

TECHNOLOGIE, KNIHOVNY
-React(Redux,Navigation,bootstrap)
-Firebase?

-navrh testů (rozvaha,ZZ,vypočítané uk)

%=========================================================================
\chapter{Implementace a testování}
%=========================================================================
\chapter{Závěr}
