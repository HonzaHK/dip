\chapter{Úvod}
Tato práce popisuje 2 související věci?. 
1)Prvním tématem je provádění finanční analýzy obecně -- shrnutí cílů a motivace, popis používaných metod, vstupních a výstupních dat, výpočtu samotného a intepretaci výsledků a omezení při dedukci?. Druhým tématem je popis programu, který takovou analýzu realizuje a který bude v rámci této diplomové práce implementován.

Finanční analýza podniku je proces, při kterém jsou získaná data (...?) podrobena rozboru a s využitím různých metod jsou vypočtena nová data s novou vypovídající hodnotou, mimo jiné i o finančním zdraví podniku. Tato data se nazývají finanční ukazatele. Lze je využít k nalezení příčin současného stavu podniku, ..., a do jisté míry? i k predikci stavu podniku v budoucnu.
různý výstup analýzy - čísla, grafy, peníze, textová forma

Text práce je rozdělen do X kapitol. Jelikož je finanční analýza poměrně rozsáhlá oblast, celá příští kapitola je věnována teorii. Nejprve je nezbytné vysvětlit specifické termíny?, jelikož se některé z nich budou v textu objevovat často a běžný čtenář? se s nimi nemusel dosud setkat. + fungování podniku, zákon , ...
Ve třetí kapitole bude popsán návrh zvolených metod vyhodnocování finančního zdraví podniku, jeho interpretace a také návrh programového řešení - architektura, implementační jazyk, formát vstupu a výstupu. Čtvrtá kapitola se zabývá realizací návrhu, implementačními detaily. V závěrečné kapitole jsou shrnuty a zhodnoceny dosažené výsledky, přínosy této práce a návrhy na její možná rozšíření.

Pro lepší demonstraci budou v této práci analyzovány výsledky hospodaření společnosti Krajči plus s.r.o., která se zabývá... Výsledky hospodaření lze najít ... .

%===========================================================================
\chapter{Problematika finanční analýzy}

vykaz zisku a ztraty + rozvaha - z ucetni osnovy na konci roku
vynosy, naklady X prijmy, vydaje (to jsou uz primo penize)
hruby\_zisk=vynosy-naklady
rozvaha=majetek firmy (at vim, k cemu pribude zisk)

Abychom mohli provést finanční analýzu, musíme mít data, ze kterých budeme vycházet. Stěžejní jsou především data o vlastním hospodaření podniku, za jejichž evidenci je zodpovědná účetní jednotka podniku, která je v průběhu roku zapisuje do tzv účtové osnovy.
\section{Účtová osnova}
Účtová osnova je 
Větší? společnosti jsou ze zákona povinny dělat? každoročně účetní závěrku. Ta má v ČR dvě povinné části - výkaz zisku a ztráty a rozvaha. 
\section{Výkaz zisku a ztráty}
\section{Rozvaha}
Narozdíl od výkazu zisku a ztráty jsou při sestavování? rozvahy data? vázána k jednomu určitému okamžiku, takzvanému rozvahovému dni, který bývá zpravidla posledním dnem účetního období.

\section{Horizontální analýza}
Určitá položka (řádek) je porovnávána se stejnou položkou za jiné časové období. Při výpočtu se tedy pohybujeme pouze v rámci daného řádku (pouze měníme rok), odtud horizontální analýza. U jednotlivých položek zvlášť je vypočtena meziroční změna a její procentuální vyjádření. Tento postup je proti ostatním velmi? přímočarý?.


\section{Ukazatele}
Existuje několik různých ukazatelů. Různé ukazatele slouží různým účelům, různé společnosti se řídí různými ukazateli. Ukazatele se počítají z výsledovky a rozvahy. Většina ukazatelů je vyjádřena číslem nebo poměrem?.
\subsection{Rentabilita aktiv}
\subsection{Vnitřní výnosové procento}

\section{Interpretace výsledků}
Vzorce a postupy pro výpočet jednotlivých ukazatelů jsou jasně (matematicky) dané. Ukazatel je však jen číslo a problémem tak může být jeho interpretace -- v některých případech nelze jednoznačně určit příčinu, důsledek nebo obecně význam jeho hodnoty. 

\section{Existující software}
\subsection{Equanta\sffamily\textregistered}

%===========================================================================
\chapter{Návrh řešení}
\section{Vybrané metody vyhodnocování finančního zdraví}
Ukazatele jsou ve skutečnosti jen čísla, kterým rozumí spíše ekonomové - běžný uživatel by z nich nejspíš nic nevydedukoval. Proto je žádoucí, aby byly výstupem nejen tato čísla, ale i nějaký srozumitelný závěr?.
Porovnání s minulým rokem
Porovnání s ukazateli v odvětví

\section{Návrh programového řešení}
Program bude mít webové rozhraní, což má samozřejmě své výhody - uživatel může k programu přistupovat odkudkoliv (i z různých platforem), používání programu není podmíněno jeho instalací a z pohledu vývojáře je mnohem jednodušší jeho údržba (vydávání nových verzí).

%===========================================================================
\chapter{Implementace a testování}
%===========================================================================
\chapter{Závěr}
